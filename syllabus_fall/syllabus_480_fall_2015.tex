\documentclass{article}
\usepackage{graphicx}
\usepackage{xcolor}
\usepackage[hidelinks]{hyperref}
\begin{document}
\begin{center}
CLARION UNIVERSITY OF PENNSYLVANIA\\
COLLEGE OF BUSINESS ADMINISTRATION\\
DEPARTMENT OF FINANCE
\\
{\bf Multinational Finance}\\
{\bf FIN 480}\\
{\bf Fall 2015}\\
\end{center}
\vspace*{6pt}
{\bf Instructor}: Matthew Brigida, Ph.D.\\
{\bf Office}: Still Hall 318\\
{\bf Office Hours}:  In Still Hall Office:  Monday 2pm--5pm, Tuesday and Thursday 3:15pm--4:15pm\\
{\bf Email}: \href{mailto:mbrigida@clarion.edu}{\textcolor{blue}{mbrigida@clarion.edu}} or \href{mailto:matt@complete-markets.com}{\textcolor{blue}{matt@complete-markets.com}} \\
{\bf Course Day/Time:} Monday 5:00--7:40 PM \\
{\bf Course Location:} Still 206 \\
{\bf Text:} International Financial Management, by Jeff Madura, 10 ed., ISBN-13: 978-1439038338 \\
\begin{center}
{\bf COURSE DESCRIPTION}
\end{center}  
An introductory survey to the fundamental principles of multinational financial management. The learning outcomes for this course are summarized below:
\begin{enumerate}
\item Understanding the determination of exchange rates under varying present regimes.  Further students will be briefly exposed to the evolution of money and foreign exchange over the past century.       
\item Introduce the international monetary system as well as the reasoning, method, and effect of central bank intervention in foreign exchange and money markets.
\item Enable students to operate firms in a multinational setting through an understanding of foreign exchange and derivative markets, and how these securities may be used to mitigate risk for the multinational firm.
\end{enumerate}
\begin{center}
{\bf ACADEMIC HONESTY POLICY}
\end{center} 
Academic dishonesty will not be tolerated in this class. Cheating
on quizzes, examinations, and other forms of dishonesty (e.g., plagiarism, collusion, and
falsification of data) will be dealt with in a serious and formal manner. The penalty for academic
dishonesty in this class will be course failure. That is, any student who is found to be cheating
or engaged in other academically dishonest behavior will be failed for this course for this
semester. Course withdrawals to avoid such a failure will not be permitted. As a student, you
have a responsibility to become familiar with the Academic Honesty Policy found in the Student
Rights, Regulations, and Procedures Handbook.\\
\begin{center}
{\bf BSBA LEARNING GOALS AND OBJECTIVES}
\end{center} 
\begin{itemize}
\item {\bf Goal 1.0: Demonstrate Business Disciplinary Competence.}  How assessed:  The exams and homeworks will evaluate a core area of finance: Operating a firm in a multinational setting. 
\item {\bf Goal 3.0 (Objectives 3.1 and 3.2):  Communicate Effectively Orally and in Written Form.}  How assessed:  The presentation of a student created Excel spreadsheet to calculate arbitrage profits given various currency call and put option prices.
\item {\bf Goal 4.0 (Objectives 4.1 and 4.3): Demonstrate Analytical Thinking Skills.} How assessed:  Students will interpret conditions in the currency, interest rate, and swap derivative markets to identify when riskless profits can be made.  If such conditions are present, students will formulate methods of obtaining those profits.
\item {\bf Goal 5.0: Understand Global Issues in the Functional Areas of Business.} How assessed:  Understanding issues in operating a Multinational Corporation such as managing currency risk exposure.  This understanding is evaluated through the exams and homeworks.
\item {\bf Goal 6.0 (Objectives 6.1 and 6.3):  Demonstrate Effective Use of Technology and Data Analysis.} How assessed:  In both homeworks and the presentation, students will analyze data and communicate conclusions using Excel.
\end{itemize}
\begin{center}
{\bf EXAMS}
\end{center} 
There will be 3 exams; two during the semester and a final exam.  All exams will be multiple choice and closed-book.  The final exam is comprehensive.  \\
\\
Normally no make-up exams will be given.  Failure to take an exam will result in a grade of zero for the missed exam.  Make-up exams will only be allowed for extraordinary and verifiable reasons.
\\
\begin{center}
{\bf HOMEWORK}
\end{center}
Three homework assignments will be assigned during the semester.   The three homework assignments will be due the week before each exam. Each homework will be worth 3 and 1/3 final grade points.  Late homework will not be accepted.
\\
\begin{center}
{\bf PROJECT}
\end{center}
Students will create a Shiny interactive web application.  To do so you'll first need to sign up for a free \href{https://www.shinyapps.io/}{\textcolor{blue}{shinyapps}} account.  

You are free to create the account under a pseudonym, so no one can publicly identify you as the owner of the account.  However, the web application is a useful tool to show off your work, and is something that can go on your resume (with a link to the application).  So you may prefer to use your real name.  My user name is 'mattbrigida'.  

Your application should have something to do with currency markets, and should be at least somewhat original.  See a gallery of applications here:  \href{http://shiny.rstudio.com/}{\textcolor{blue}{shiny.rstudio}}. Possible applications may be:
\begin{itemize}
\item Plot a time series of a currency's direct and indirect quotes.
\item Plot a time series of many currencies.
\item A currency converter.
\item A cross rate converter.
\item An arbitrage calculator (locational, triangular, covered interest, etc).
\item Create a histogram or probability density plot for currency returns.
\end{itemize}
To get started you will want to use the RStudio development environment for R.  This is available in the Still hall computer lab, or you can install it for free on your own computer from here:  \href{https://www.rstudio.com/products/rstudio/download/}{\textcolor{blue}{download}}.  If you install it on your own computer you'll need to install R first.  You can get R here:  \href{https://cran.r-project.org/}{\textcolor{blue}{download}}
% Students will create forward contracts through the use of currency options, and calculate arbitrage profits given various options prices.  In so doing, students will gain an understanding of how various derivative markets relate to each other, and also how derivative markets relate to the underlying spot market.  
\\
% \begin{center}
% {\bf CURRENCY ANALYSIS}
% \end{center}
% Each student will submit an analysis of currency markets.  The analysis should be about a page long (double spaced) and cover what is driving recent movements in the main currencies.  The analysis may also include projections of future currency movements.  I may post some analyses to D2L for your classmates to read.  We may also discuss the currency analyses in class - so you should be prepared to discuss your analysis.  Lastly, the most important requirement of your analysis is that it be your original work.  {\bf Plagiarizing any part of your analysis will result in a zero for the entire analysis.}  Use your own words.
% \\
\begin{center}
{\bf PARTICIPATION/QUIZZES/COMPUTER ASSIGNMENTS}
\end{center}
Throughout the semester I may give pop quizzes in class, and assign short homeworks (about 5) that mainly involve downloading data from various internet sources (FRED, Yahoo Finance, Oanda, etc.) and performing various calculations or graphically presenting the data.  Each quiz and assignment will be weighted equally. During the semester, I may increase the amount of participation points.
\\
\begin{center}
{\bf COURSE COMMUNICATION}
\end{center}  
All important/official announcements will be posted on Desire 2 Learn and emailed to each student's Clarion University email account.  I will post helpful information to: \href{http://www.complete-markets.com}{\textcolor{blue}{Complete Markets}}.  To see information relating to your course type ``FIN 480'' in the search bar at the upper left of the web page.  Some examples of helpful information are spreadsheets which assist in studying for exams or completing homeworks, answers to questions other students have asked (of course I will not include who asked the question), and useful \href{http://www.r-project.org}{\textcolor{blue}{R}} code.
\\
\begin{center}
{\bf GRADING}:\\
\end{center}
Exam 1 ..................................................   20\\
Exam 2 ..................................................   20\\
Final Exam ............................................  20\\
Homework .............................................   10\\
Project .................................................    20\\
% Currency Analysis .................................   10\\
Participation/Computer Assignments ...  10\\
Total Points ..........................................  100\\
\\
\begin{center}
{\bf Final grades will be assigned according to the following scale}:
\end{center}
\begin{itemize}
\item 90 - 100 A
\item 80 - 89.9 B
\item 70 - 79.9 C
\item 60 - 69.9 D
\item $<$ 60 F
\end{itemize}
\vspace*{5pt}
\begin{center}
{\bf An Important Note on Grading}
\end{center}
There is no special consideration if you need a certain grade in this course to
graduate. {\bf If you require a certain grade in this class to graduate it
is your responsibility to earn that grade.} Specifically if you receive a `D'
in this course I will not allow you to do extra assignments after the course is
complete in exchange for a higher grade.
\\
\begin{center}
{\bf GENERAL NOTES}:
\end{center}
\begin{enumerate}
\item Attending class and reading the text is required.
\item All exams will be closed book.
\item There will be no make up exams or extra points assignments.
\item If you are late for an exam, no extra time will be allotted to you.
\item You should bring your text to class.
\item You are expected to be on time for class. This is especially important for exam
dates.
\item Disruptive behavior in the classroom will not be tolerated.
\item You may not use tobacco products in class.
\item Cheating will result in prosecution to the fullest extent possible under university rules.
\item You are responsible for material covered in the lectures, as well as text material.
\item  {\bf Adding or Dropping the Course:} To add or drop the course the student should consult the university guidelines and withdrawal dates.
The course instructor is not involved in a student's adding or withdrawing from the course.
\item {\bf Software:} You may want to use word processing and spreadsheet software in
this course. Common examples of such software are Microsoft Word and
Excel. However, there is no need to buy this software if you don't already
have it. There are many free (open-source) alternatives which are just as
good (and which allow you to save/read files as .doc(x), .pdf, and .xls(x)).
Some widely used free office suites are \href{http://www.libreoffice.org}{\textcolor{blue}{LibreOffice}}
and  \href{http://www.openoffice.org}{\textcolor{blue}{OpenOffice}}. Feel free to download and use
these. {\it All word processed submissions should be in .pdf, and
all spreadsheets should be submitted as .xlsx.}
\end{enumerate}
\vspace*{5pt}
\begin{center}
{\bf TENTATIVE OUTLINE}
\end{center}
8/24:  Chapter 1 \\
8/31:  Chapter 2\\
9/7: Chapter 3\\
9/14: Chapter 4 \\
9/21:  Homework and exam review \\
{\bf \textcolor{red}{9/28:  Exam 1}} \\
10/5: Chapter 5\\
10/12: Chapter 6\\
10/19:  Chapter 7\\ 
10/26:  Homework and exam review \\ 
{\bf \textcolor{red}{11/2:  Exam 2}}\\
11/9: Chapter 8\\
11/16: Chapter 10\\
11/30: Chapter 11, and exam review.\\

\clearpage

\begin{center}
{\bf Statement Required by PASSHE}  
\end{center}

Clarion University and its faculty are committed to assuring a safe and productive educational environment for all students. In order to meet this commitment and to comply with Title IX of the Education Amendments of 1972 and guidance from  the Office for Civil Rights, the University requires faculty members to report incidents of sexual violence shared by students to the University's Title IX Coordinator.                         

The only exceptions to the faculty member's reporting obligation are when incidents of sexual violence are communicated by a student during a classroom discussion, in a writing assignment for a class, or as part of a University-approved research project.                           

Faculty members are obligated to report sexual violence or any other abuse of a student who was, or is, a child (a person under 18 years of age) when the abuse allegedly occurred to the person designated in the University protection of minors policy. 


\end{document}

%%% Local Variables:
%%% mode: latex
%%% TeX-master: t
%%% End:
